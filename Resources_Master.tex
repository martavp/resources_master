\documentclass[3p]{elsarticle} % seleccionar: preprint, review, 1p, 3p, 5p
% Add following lines to remove the footnote "preprint submited to"
\makeatletter
\def\ps@pprintTitle{%
 \let\@oddhead\@empty
 \let\@evenhead\@empty
 \def\@oddfoot{}%
 \let\@evenfoot\@oddfoot}
\makeatother

% Add following two lines to use Arial
\usepackage{helvet}
\renewcommand{\familydefault}{\sfdefault}

% Add following line to get figures capation as Figure S1
%\renewcommand{\thefigure}{S\arabic{figure}}
%\renewcommand{\thetable}{S\arabic{table}}
%\renewcommand{\thesubsection}{S\arabic{subsection}}
%\renewcommand{\refname}{Supplemental References}

\usepackage{mathtools}
\journal{ }


%to force all images and table in one single section
\usepackage{placeins}
% It is necesary to add \FloatBarrier in the text. 
% After that order, all the floating are shown.

%%%%%%%%%%%%%%%%%%%%%%%
%% Elsevier bibliography styles
%%%%%%%%%%%%%%%%%%%%%%%
%% To change the style, put a % in front of the second line of the current style and
%% remove the % from the second line of the style you would like to use.
%%%%%%%%%%%%%%%%%%%%%%%

%% Numbered
%\bibliographystyle{model1-num-names}

%% Numbered without titles
%\bibliographystyle{model1a-num-names}

%% Harvard
%\bibliographystyle{model2-names.bst}\biboptions{authoryear}

%% Vancouver numbered
%\usepackage{numcompress}\bibliographystyle{model3-num-names}

%% Vancouver name/year
%\usepackage{numcompress}\bibliographystyle{model4-names}\biboptions{authoryear}

%% APA style
%\bibliographystyle{model5-names}\biboptions{authoryear}

%% AMA style
%\usepackage{numcompress}\bibliographystyle{model6-num-names}

%% `Elsevier LaTeX' style
\bibliographystyle{elsarticle-num}

%%%%%%%%%%%%%%%%%%%%%%%
\hyphenation{}
\usepackage{eurosym}
\usepackage{threeparttable} % allow the use of footnote within tables

\usepackage{url}
\usepackage[colorlinks=true, citecolor=blue, linkcolor=blue, filecolor=blue,urlcolor=blue]{hyperref}

%to add the number to the lines
\usepackage{lineno}
\modulolinenumbers[5]

\usepackage{lineno,hyperref}
\modulolinenumbers[1]
\usepackage{amsmath}
\usepackage{siunitx}
\usepackage{eurosym}
\biboptions{numbers,sort&compress}
\usepackage[europeanresistors,americaninductors]{circuitikz}
\usepackage{adjustbox}
\usepackage{xspace}
\usepackage{caption}
\usepackage{booktabs}
\usepackage{tabularx}
\usepackage{threeparttable}
\usepackage{multicol}
\usepackage{float}
\usepackage{graphicx,dblfloatfix}
\usepackage{csvsimple}
\usepackage{amsmath}
%% new commands
\newcommand{\ubar}[1]{\text{\b{$#1$}}}
\newcommand*\OK{\ding{51}}
%\renewcommand*\nompostamble{\end{multicols}}
\newcommand{\specialcell}[2][c]{%
	\begin{tabular}[#1]{@{}l@{}}#2\end{tabular}}
\newcommand{\ra}[1]{\renewcommand{\arraystretch}{#1}}	

\def\co{CO${}_2$}
\def\el{${}_{\textrm{el}}$}
\def\th{${}_{\textrm{th}}$}


%\renewcommand*{\today}{July, 10 2018}
%\hypersetup{draft} %to avoid problems with hyperref while drafting

\begin{document}

\begin{frontmatter}
\title{Recommended resources for writing your MSc project}

\author[mymainaddress]{Marta Victoria}
\address[mymainaddress]{\href{mvipe@dtu.dk}{mvipe@dtu.dk}, Department of Wind and Energy Systems, Technical University of Denmark, Lyngby, Denmark}

%\begin{abstract}

%\end{abstract}

%\begin{keyword}

%\texttt{elsarticle.cls}\sep \LaTeX\sep Elsevier \sep template
%\MSC[2010] 00-01\sep  99-00
%\end{keyword}

\end{frontmatter}

\pagenumbering{gobble} % supress page numbers

This document contains suggestions and links to resources that might be useful for performing and writing your MSc thesis. 
You might also find them useful for writing the special-course report.

\section{Existing literature on a research topic}

To search for scientific articles on a research topic the following search engines are recommended.

\begin{itemize}
\item \href{https://scholar.google.com/}{https://scholar.google.com/}
\item \href{https://www.sciencedirect.com/}{https://www.sciencedirect.com/}
\end{itemize}

This \href{https://docs.google.com/document/d/1pZOAneNj5jnTuIJzg81mwKVoQLWyduYcxtk2No2SzSk/edit?tab=t.0}{document} provides suggestions on how to read scientific papers. 

\section{Writing style}

\subsection{Style}
Similarly to a scientific article, a master project must include the following sections:

\subsubsection{Summary}
It should be about 1-page long. It should include a description of the context of why your project is relevant, the main research question, a brief mention of the methods, and a summary of your results.  This is the part of your document that most people will read, so make sure not to write it on the last minute, share it with colleagues and friends, and ask for feedback. 

\subsubsection{Introduction}

Make sure to include the following elements:
\begin{itemize}
\item[a)]  Motivation for this project: Why is this research interesting? Why should the reader care about this? 

\item[b)]  State of the art or previous relevant literature. Notice that this should be written in a way that naturally leads to (c) and (d). For example, when mentioning previous papers, make sure to state the limitations that your novelty (d) will overcome.

\item[c)]  What is/are the main research question/s that you are trying to address?

\item[d)]   What is the main novelty of your project? What are you doing that has not been done before?
\end{itemize}


\subsubsection{Methodology (Modelling or experimental methods) }
Try to be concise and mention all the information that is relevant to understand the results. When using equations, make sure the define all the symbols contained in them. 
You can include subsections, e.g. one describing the relevant theory (including additional theoretical background, possibly referring to literature) and other/s describing the implementation (data, set-up, equations, code).

\subsubsection{Results}
You can start structuring this section by selecting the 5 or 6 most relevant figures. What is the key message that you want to communicate in each of them? You can also structure the results into subsections. 




\subsubsection{Discussion}
Make sure you include the following elements:
\begin{itemize}

\item[a)] How do your results compare to previous literature, do they confirm or contradict previous results?
\item[b)] What are the policy implications of your results? What do they mean for someone who is thinking about regulation?
\item[c)] What are the implications for energy system modelling? What should we do better?
\item[d)] What are the limitations of your approach/paper? Be realistic and honest, this can sometimes prevent certain comments from the examiners. Can also be connected to the outlook/future research directions.
\end{itemize}

\subsubsection{Conclusions}
Think about take-away messages and what the relevance of this paper for society and the scientific field can be. Make sure that you answer the research question that you posed in the introduction.

\subsubsection{Outlook}
Think about some of the limitations that you listed before. Also, what would you do if you had more time to work on this project?

\subsubsection{Bibliography}

\

\subsubsection{Code in the document}
It is a very good idea to refer to your \href{https:\\github.com}{Github} repository and codebase in the thesis document itself, with a short description on how it’s reproducible even if this might be in the \textit{README.md} file. This is considered good practice in open science, and might be helpful if somebody reads the thesis and does not look at the GitHub repository.

%This reference \cite{Socolofsky_2004} nicely describes what is the purpose of each section and what should be included in each of them. \\

\subsection{Figures}

Make sure that every figure has a caption, every axis has a title, and the font size in the figure is similar to that in the text. This \href{https://doi.org/10.1371/journal.pcbi.1003833}{paper} provides ten simple rules to create better scientific figures. The \href{https://www.ipcc.ch/site/assets/uploads/2019/04/IPCC-visual-style-guide.pdf}{IPCC Visual Style Guide} also includes a lot of good recommendations. \href{https://www.addtwodigital.com/blog}{addtwodigital} and \href{https://github.com/cxli233/FriendsDontLetFriends}{The FriendsDontLetFriends repository} include a good list of suggestions for figures and common mistakes in data visualization to be avoided. 

\subsection{Template}

It is highly recommended to write the manuscript using Latex. You might want to use the online tool \href{https://es.overleaf.com}{overleaf} that allows writing shared tex documents and it is very intuitive.\\

There is no official template for Master thesis document at DTU, but you can use e.g. \href{https://www.overleaf.com/latex/templates/dtu-thesis-template/dyxwwkhmzrbx}{this one}. Notice that the Study Board will send you via email an  official cover in pdf that must be included in your MSc thesis. 

%You can find a template for the Master project in \href{https://github.com/martavp/resources_master/tree/master/LatexTemplate_MasterThesisProject}{this folder}, which is known to work in overleaf. The use of this particular template is not mandatory.

\subsection{Language}

Structure your writing. Follow the rule: One paragraph-One main idea. Read your paragraphs again and check: What is the main idea that you are trying to communicate? Is it properly done in the current paragraph?

\href{https://app.grammarly.com/}{Grammarly} is a nice tool to check your text and avoid spelling errors. \\

For Danish speakers, this \href{https://www.youtube.com/watch?v=I_i9bvv3N4M}{video} summarizes the key rules on how to use commas in English. 



\section{Add a license to your Master thesis and code.}
 
\textbf{Why?} A license clarifies the conditions under which your code, text, data, or figures can be reused. In the absence of a license, the author still retains proprietary copyright, and the conditions under which the materials can be used are unclear. You want other people to use your work, the best way is to create the conditions so that they can do it. \\

\textbf{How?} You can select a license using \href{https://choosealicense.com/}{https://choosealicense.com/}. For a master thesis, the easiest is to use \href{https://creativecommons.org/licenses/by/4.0/}{Creative Commons Attribution 4.0 International}. To do that you need to add the following sentence to one of the first pages of your document
\begin{quote}

\textit{Copyright  \textcopyright  2026 *add your name* $<$ *add your email* $>$ \\
This work is licensed under a Creative Commons Attribution 4.0 International Licence (CC-BY).}

\end{quote}

\section{Example of Master thesis document from previous students}
You can find some previous documents \href{https://dtudk.sharepoint.com/:f:/r/sites/ResilientEnergySystems/Delte\%20dokumenter/General/PhD,\%20MSc\%20thesis\%20repository/Master\%20thesis/DTU?csf=1&web=1&e=fVmT0R}{in this folder.}


\section{Open Energy Modelling}

The Open Energy Modelling Initiative gathers Master, PhD Students and other researchers dealing with energy modelling worldwide. It is a wonderful platform to learn what others are doing and what has been done in the past. To search on a particular topic you can: (a) register \href{https://forum.openmod-initiative.org/}{in the forum} and search by topic, (b) subscribe to the \href{https://groups.google.com/forum/#!forum/openmod-initiative/home}{distribution list} where you can also search on previous topics or (c) look in the \href{https://wiki.openmod-initiative.org/wiki}{wiki}. 

\section{Python, PyPSA and PyPSA-Eur}

If you are using PyPSA or PyPSA-Eur for your project, check this \href{https://docs.google.com/document/d/1304ZUwydXr0ks1uuM7TJCyYcRtMPdB2h/edit}{collaborative document on how to learn PyPSA-Eur}. 


\subsection{Coding}

It is highly recommended to write high-quality code with proper levels of documentation so that the analyses that you perform can be understood, used and extended by others. Most probably, you will benefit the most from it when you try to read you own code some weeks after writing it. \\

\href{https://www.python.org/dev/peps/pep-0008/}{PEP-8} is a standard to write easy-to-read python code. Take a look at it and try to follow the rules in PEP-8.

\subsection{Sharing code}

To write code in a collaborative way, researchers use repositories where improvements can be progressively added. A nice platform that host public and private repositories is \href{https:\\github.com}{Github}. In short, every user has its own local copy of the repository where he/she makes changes and push them to the remote repository shared by the collaborators. You can create your own repository to share your project with others and allow them to contribute to it. Similarly, you can contribute to projects by other researchers.
\\\href{https://github.com/martavp/integrated-energy-grids/blob/main/Lectures/IEG_LECTURE13_Open\%20Science.pdf}{Slides 19-24 on this lecture} provide a short summary on how to use Github. 


%\bibliography{bibliography}

\end{document}